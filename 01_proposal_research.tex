\documentclass[11pt, letterpaper, titlepage]{article}

% Compact lists
\usepackage{enumitem}

% Images
\usepackage{graphicx}

% Tables
\usepackage{booktabs}

% Document boarders
\usepackage{geometry}
\geometry{
    letterpaper,
    margin=.85in
}

% Remove page numbers
% \pagenumbering{gobble}

% Double spacing
\usepackage{setspace}
% \doublespacing

% Spacing out footnotes
\setlength{\footnotesep}{0.75\baselineskip}

% Palatino
\usepackage{newpxtext}
\usepackage{newpxmath}

% Nicer monospace font
\usepackage{inconsolata}

% Improved typography
\usepackage{microtype}

% URL and hyperlink settings
% \usepackage[dvipsnames]{xcolor}
\usepackage[table, dvipsnames]{xcolor}
\usepackage{xurl}
\usepackage[
    colorlinks,
    breaklinks=true,
    urlcolor=NavyBlue,
    citecolor=NavyBlue,
    linkcolor=NavyBlue,
    % breaklinks,
    linktocpage
]{hyperref}
\urlstyle{same}

% Bib settings
\usepackage[
    backend=biber,
    % style=apa,
    style=chem-acs,
    doi=true,
    url=true,
    % url=false,
    articletitle=true,
    maxbibnames=99
    %natbib=true
    ]{biblatex}
\addbibresource{bibliography.bib}

% \usepackage[
%     backend=biber,
%     % style=apa,
%     style=chem-acs,
%     doi=true,
%     url=true,
%     articletitle=true,
%     chaptertitle=true,
%     maxbibnames=99
%     %natbib=true
% ]{biblatex}
\newcommand{\citean}[1]{\citeauthor{#1}\autocite{#1}}

% % ToC Settings
\usepackage{tocloft}
% Dots for sections
\renewcommand{\cftsecleader}{\cftdotfill{\cftdotsep}}
% No dots for other sections
\renewcommand{\cftsubsecdotsep}{\cftnodots}
\renewcommand{\cftsubsubsecdotsep}{\cftnodots}

% Spacing for ToC section titles
\setlength{\cftsubsecnumwidth}{0pt}
\setlength{\cftsubsubsecnumwidth}{0pt}

% Custom commands for new section titles to TOC
\newcommand{\custsubsection}[1]{%
    \subsection*{\thesubsection {#1}} %
    \addcontentsline{toc}{subsection}{\protect\numberline{}{\thesubsection #1}} %
    \addtocounter{subsection}{1}
}

\newcommand{\custsubsubsection}[1]{%
    \subsubsection*{\thesubsubsection {#1}} %
    \addcontentsline{toc}{subsubsection}{\protect\numberline{}{\thesubsubsection #1}} %
    \addtocounter{subsubsection}{1}
}

% Section numbering
% \newcommand{\secref}[1]{\S\ref{#1}}
% \newcommand{\secrefdot}[1]{\ref{#1}}
\newcommand{\secref}[1]{\hyperref[#1]{\S\ref{#1}}}
% \newcommand{\secrefdot}[1]{\hyperref[#1]{\ref{#1}.}}
\newcommand{\secrefboth}[1]{\hyperref[#1]{\ref{#1}.} & \nameref*{#1}}

\newcommand{\ditto}{\multicolumn{1}{c}{\texttt{"}}}

\begin{document}

\begin{titlepage}
    \begin{center}
        % \vspace*{250px}
        \vspace*{2cm}
        \Huge
        Machine Learning Algorithm Development for Quality Assurance\\of Lutz Catchment Run-Off Data
        
        \bigskip

        \LARGE
        Project Proposal \& Statement of Work
        
        \bigskip

        Gillian McGinnis, Project Manager
        
        \bigskip
        % List remaining team members here, one name per line, first name, last name, alphabetically by last name
        
        POTENTIAL ADVISORS:\\ %\emph{LIST POTENTIAL ADVISORS FOR GROUP HERE}
        \Large
        Dr. Sriram Iyengar (\emph{The University of Arizona College of Medicine, Phoenix}),\\
        Steven Paton (\emph{Smithsonian Tropical Research Institute})
        % Dr. Steven Paton (\emph{Smithsonian Tropical Research Institute})
        
        \bigskip

        \LARGE
        Date: September 23, 2025

        \bigskip
        %     \end{center}
% \end{titlepage}

% \begin{titlepage}
%     \title{Machine Learning Algorithm Development for Quality Assurance\\of Lutz Catchment Run-Off Data}
%     \author{Gillian McGinnis, Project Manager}
%     % \date{}

%     \maketitle
% \end{titlepage}

% Use the statements in red italics to help you write this document. Delete all the red italics when you are done.
% This document is both a project proposal and statement of work document.  It states, “Here’s the problem we have identified and this is our plan to develop a solution.”  It also clearly details what you will require from the mentor to be successful. It is the contract between you, and the mentor which summarizes who will do what and the requirements for success.
% SOW: The Product Specification defines your product in detail. What you are making, how it works, how it is used and how it will be tested.

% Each section shall have a named author, but all team members are responsible for the overall document.  Meaning one team member writes the section and the other team members review it for errors.  List the author of each section immediately following the heading and in the author table in the grading rubric at the end. Example: “This section was written by John Doe.”  It is the team and each members’ responsibility to divide the document equitably. Individual scores will be determined by the quality and quantity of each person’s work.
% Remember the goal is to convey all the information – not applicable because…. is relevant information.  Use bulleted lists and white space smartly. Include figures and tables as required and necessary. For each table and figure, write a brief introductory sentence “As shown in Table 2 five different pages are used…”  Put the detail in the figures and tables, keep the paragraphs short and do not duplicate information needlessly.  Your goal is to convey the full amount of information in the shortest possible document. 
% You need to maintain a Revision History Table like this: (Not this exact table with ONLY the dates changed)


        \normalsize
        % Some text.
        \begin{table}[h!]
            \centering
            \begin{tabular}{l l l} 
                \toprule
                Version & Summary of Changes & Date \\
                \midrule
                % 0.1 & Converted template from \texttt{.docx} to \LaTeX & 09/13/25 \\
                % 0.2 & Started bibliography, \secref{sec_es} Executive Summary & 09/13/25 \\
                % 0.3 & Began User Research, Features; Transcribed Project Timeline; finished Ethics & 09/16/25 \\
                0.1 & Converted template to \LaTeX; started \hyperref[sec_bib]{References}, \secref{sec_es} & 09/13/25 \\
                % 0.2 & Started bibliography and \secref{sec_es} & 09/13/25 \\
                % 0.2 & Began \secref{sec_market}, \secref{sec_features}; transcribed \secref{sec_time}; completed \secref{sec_ethics} & 09/16/25 \\
                % 0.3 & Continued \secref{sec_features}; updated \secref{sec_time} & 09/18/25 \\
                0.2 & Began \secref{sec_lit}; transcribed \secref{sec_time}; completed \secref{sec_ethics} & 09/16/25 \\
                0.3 & Continued \secref{sec_lit}; updated \secref{sec_time} & 09/18/25 \\
                0.4 & Switch to research template; updated \secref{sec_lit} & 09/19/25 \\
                0.5 & Updated \secref{sec_time}, expanded \secref{sec_deliv}; added \hyperref[fig_gantt]{Gantt chart} & 09/22/25 \\
                \bottomrule
            \end{tabular}
        \end{table}
        % \begin{table}[h!]
        %     % \centering
        %     % \caption{An example of the raw spectra data. The ``Variety'' column is not unique for the dataset, and can be disregarded during analysis.}
        %     % \label{tab_revision}
        %     \begin{tabular}{l l l} 
        %         \toprule
        %         Version & Summary of Changes & Date
        %         \midrule
        %         % 0.1 & Transferred template from .docx to \LaTeX & 09/13/25 \\
        %         V7 & Added Sponsor and ADvisor Requirements & 8/17/22
        %         \bottomrule
        %     \end{tabular}
        % \end{table}

    \end{center}
\end{titlepage}

% The revision history table documents major revisions to the document. In addition to showing the evolution of the project it also provides a way for reviewers to quickly see which sections have changed so they do not have to reread the entire document each time. It is a high-level summary, minor edits, spelling corrections, etc. should not need to be summarized in this block. It is your responsibility to determine which sections have significant changes that merit review and decide when it makes sense to declare a revision number (hint: call a team meeting to review it page by page). Less detail is required for early revisions.  Typically, teams use collaborative tools like Google Docs or Office 365 to allow real-time editing. There may be hundreds of changes made in a few days; the revision block should reflect only the high-level changes. E.g., “Section 2.1 was updated to reflect the switch to the xyz processor.” For a shorter document such as the SOW you may only have 2 or 3 versions.  If you do update it later be sure to include the revision information.
% The table of contents below is auto generated – as long as you use Heading 1 and Heading 2 styles you should not have to type anything into the table.  You will have to refresh it.  The pagination on this template is not set, use your best judgement on when to start a new page for clarity balanced with keeping excessive white space out of your document.  A suggestion is to have the revision block and TOC on one page, insert a page break before the Executive Summary, then do what make sense and looks professional after that.



\tableofcontents

\pagebreak

% How does this work \autocite{sasu2025,srdrRepo}?

\section{Executive Summary}\label{sec_es}
% The Executive Summary was written by John Doe. (Complete this with the author of each section)

% The executive summary is a short review of the important aspects of the project.  It is very similar to the elevator pitch, but it is slightly longer with more detail. This section should be about 3 paragraphs and about ¾ of a page.

% 1st paragraph - Introduce your project - What is your research project and its key features? (“Our product [is a high efficiency controller/web-based application/wireless smart sensor/….] that [displays/computes/transforms/…] [temperature data/DC current/] into [easy to read graphs/control signals/wireless data] while [lowering cost/improving response time/achieving high efficiency/].
The primary goal of this project is to create models that can automate quality assurance of temporal run-off data from the Lutz~Catchment of Barro~Colorado~Island, Panama (Figure~\ref{fig_map}).
% The algorithms will conduct quality assurance of temporal run-off data from the Lutz~Weir of Barro~Colorado~Island, Panama.
This will remove the need for manual data corrections, which not only use the valuable time and energy of researchers, but also have the potential to be imprecise or inconsistent.
% This will streamline the analytical process for researchers, freeing up more time and energy for more pressing analyses.
Rainfall and soil moisture measurement data can be integrated to produce a more well-informed model.
% Other variables---such as rainfall and soil moisture measurements---can be integrated to produce a more well-informed model.

% 2nd paragraph - Explain why the project is needed, what problem it is solving. If current solutions to the problem exists explain how your solution is better and unique. Some context is needed here but only enough to establish “Why?” product will benefit the user.) If it is an internal project, state why it is important. This may be to facilitate learning, but hopefully it also has a tie to some greater good for society. Find credible sources and cite them in a footnote. Don’t say “everyone knows flooding is an issue…” research and say, “according to <insert credible reference, peer reviewed paper, gov’t agency, etc> flooding in 2019 caused $32M in damages in Acme County, our product will aid in the detection….”
% The Lutz~Weir (Figure~\ref{fig_weir}) is located on the northeast slope of Barro~Colorado~Island, Panama, and has been collecting run-off data since January 1, 1972.
% The Lutz~Weir is located on the northeast slope of Barro~Colorado~Island, Panama (Figures~\ref{fig_map},~\ref{fig_weir}).
% The Lutz~Weir is located on the northeast slope of Barro~Colorado~Island, Panama (Figure~\ref{fig_map}).

The weir has been electronically collecting run-off data since 1989 in frequencies ranging from three~minutes to three~hours, resulting in millions of data points with more added every day.
Quality issues in the data come about due to environmental factors, sensor issues, and calibration requirements.
Currently, quality assurance is conducted manually in Visual~FoxPro interactive interface.
Not only does this approach run the risk of imprecise or inconsistent corrections, but the deprecated status of the language and interface leaves it vulnerable.
Past attempts at automated quality assurance have failed due to the failure modes' resilience against traditional stochastic methods.

% % LONGER
% The weir has been electronically collecting run-off data since July 19, 1989, with frequency measurements ranging from three~minutes to three~hours.
% % It has been collecting run-off data since January 1, 1972, with frequency of measurements ranging from three~minutes to three~hours.
% % Until July 19, 1989, water level was recorded on paper charts, which have since been digitized.
% % Frequency of measurements range from three~minutes to three~hours.
% % In addition to rainfall measurements from Clearing station 170~m north of the weir, the run-off data contains millions of records over time.
% The set contains millions of data points, with more added as the sensor continues to record measurements.
% % Millions of data points are currently present in the set, with more added as the sensor continues to record measurements.
% % 
% Quality issues in the data come about due to the need for re-calibration, debris blockages resulting in monomodal run-off value increases without rainfall, complications from weir blockages paired with rainfall, false positives, impossible sub-zero values, evapotranspiration, considerations for annual intentional pond drains, and signal noise.
% % 
% Currently, quality assurance is conducted manually via the Visual~FoxPro interactive visual interface.
% Not only does this approach run the risk of imprecise or inconsistent corrections, but the deprecated status of the language and interface leaves it vulnerable.
% % 
% % Past attempts at quality assurance using stochastic methods have failed, as past teams have attempted to automate it with stochastic methods but concluded it to be resilient against automation.
% Past attempts at automated quality assurance have failed due to the failure modes' resilience against traditional stochastic methods.
% %

% \pagebreak

% \begin{figure}[hbt!]
%     \centering
%     \fbox{\includegraphics[width=.45\textwidth]{figures/map_bci.jpg}}
%     % \caption{The Clearing (``\textit{El~Claro}'') on Barro~Colorado~Island, Panama, in relation to Panama City.~\parencite{stri2025}}
%     % \caption{The Clearing (``\textit{El~Claro}'') on Barro~Colorado~Island, Panama, where the Lutz~Weir is located.~\parencite{stri2025}}
%     \caption{The Clearing (``\textit{El~Claro}'') on Barro~Colorado~Island, Panama.\autocite{stri2025}}
%     % \caption{The location of Barro~Colorado~Island in Panama CITE.}
%     \label{fig_map}
% \end{figure}

% \begin{figure}[hbt!]
%     \centering
%     \fbox{\includegraphics[width=.5\textwidth]{figures/map_bci_soil.jpg}}
%     % \caption{The Clearing (``\textit{El~Claro}'') on Barro~Colorado~Island, Panama, in relation to Panama City.~\parencite{stri2025}}
%     % \caption{The Clearing (``\textit{El~Claro}'') on Barro~Colorado~Island, Panama, where the Lutz~Weir is located.~\parencite{stri2025}}
%     \caption{The locations of the weir and tower in relation to the ten soil moisture sampling sites in the Lutz catchment on Barro~Colorado~Island, Panama.\autocite{stri2025}}
%     % \caption{The location of Barro~Colorado~Island in Panama CITE.}
%     \label{fig_map}
% \end{figure}

\begin{figure}[!hbt]
    \centering
    \fbox{\includegraphics[width=.6\textwidth]{figures/map_bci_soil.jpg}}
    \caption{The locations of the weir and tower in relation to the ten soil moisture sampling sites in the Lutz catchment on Barro~Colorado~Island, Panama.\autocite{stri2025}}
    % \caption{The location of Barro~Colorado~Island in Panama CITE.}
    \label{fig_map}
\end{figure}

% \begin{figure}[hbt!]
%     \centering
%     \includegraphics[width=.6\textwidth]{figures/lutz_weir_wide.jpg}
%     \caption{The Lutz~Weir \parencite{stri2025}.}
%     % \caption{The Lutz~Weir CITE.}
%     \label{fig_weir}
% \end{figure}

% \pagebreak

% In addition, rainfall measurements have been recorded at the Clearing station 170~m north of the weir.

% 3rd Paragraph - Describe the process of development: What work will you be doing (“We will be /aquiring data through web scraping, and analyzing XYZ) Who is involved, Where the project work will take place, what you will have completed at the end of semester.

% Given that the project is still in its early stages, a literature review will first need to be conducted to identify comparable machine learning approaches for quality assurance of temporal data.
% Table~\ref{tab_dol}
% Fortunately, 

An outline of anticipated project responsibilities is provided in Table~\ref{tab_dol}.
Relevant data sets are available on the Smithsonian~Research~Data~Repository with \texttt{CC~BY~4.0}~licenses.\autocite{srdrRepo}.
% All data sets are readily available in the form of \texttt{.csv} files, which are regularly posted to the Smithsonian~Research~Data~Repository online repository with \texttt{CC BY 4.0} licensing.\autocite{srdrRepo}
% Relevant data will first need to be uniformly coded and united into a cohesive set in order to conduct relevant analyses.
% Include a table as shown. For this and all tables you do not want to repeat the information in the table – that’s what the table is for.  Rather you should introduce the table and point out anything significant. Something like “Table 1 shows the preliminary division of responsibilities among the team members.  Since the User Interface for this product is quite complex, we have divided it among multiple team members as listed.”
% 
% 
% An outline of anticipated project responsibilities is provided in Table~\ref{tab_dol}.
% 
% Steven~Paton---director of the Physical Monitoring Program at the Smithsonian~Tropical~Research~Institute---will be the primary contact regarding background information of the data, knowledge of exceptional data point ranges (such as extenuating circumstances, e.g., a tree fall damaging equipment), and any hydrological questions that come about.
Steven~Paton---director of the Physical Monitoring Program at the Smithsonian~Tropical~Research~Institute---will be the primary contact regarding background information of the data, knowledge of exceptional data points, and any hydrological questions that come about.
Project work will be conducted remotely, with regular advisor check-ins via teleconference.

The goals by the end of the project are to have a system for multiclass classification to flag periods of time with failure mode(s), and models to correct flagged data points.
% By the end of the project, it is anticipated that a model to flag and classify ranges of time containing a failure mode will be possible.
% Ideally, the model will also be able to flag periods with multiple/simultaneous failure modes.
% Additionally, models will be created to attempt quality assurance corrections on flagged time periods.
% Although not all failure modes may be able to be successfully addressed (for instance, there is only one major incident of signal noise), at least two should be addressed.
At least two failure modes are expected to be addressed.
Measured success of the models can determine which failure modes are the most difficult to accurately repair.

% \pagebreak

% \begin{figure}[!h!tp]
% \begin{figure}[hbt!]
%     \centering
%     \fbox{\includegraphics[width=.6\textwidth]{figures/map_bci_soil.jpg}}
%     % \caption{The Clearing (``\textit{El~Claro}'') on Barro~Colorado~Island, Panama, in relation to Panama City.~\parencite{stri2025}}
%     % \caption{The Clearing (``\textit{El~Claro}'') on Barro~Colorado~Island, Panama, where the Lutz~Weir is located.~\parencite{stri2025}}
%     \caption{The locations of the weir and tower in relation to the ten soil moisture sampling sites in the Lutz catchment on Barro~Colorado~Island, Panama.\autocite{stri2025}}
%     % \caption{The location of Barro~Colorado~Island in Panama CITE.}
%     \label{fig_map}
% \end{figure}

% \begin{table}[!h!bp]
\begin{table}[htbp]
    \centering
    \begin{tabular}{ll}
        \toprule
        Team Member & Feature responsibility \\
        \midrule
        Gillian McGinnis & Literature review \& research \\
        \ditto & Exploratory data analysis (EDA) \\
        \ditto & Failure mode classification model \\
        \ditto & Failure mode correction models \\
        \ditto & Model quality evaluations \\
        \ditto & Reproducibility documentation \\
        \ditto & Course deliverables and presentation preparations \\
        % \ditto & User interface (\textit{time permitting}) \\
        \bottomrule
    \end{tabular}
    \caption{Preliminary Subsystem Responsibilities}
    \label{tab_dol}
\end{table}

% By the end of the project, it is anticipated that a model to flag and classify ranges of time containing a failure mode will be possible.
% Ideally, the model will also be able to flag periods with multiple/simultaneous failure modes.
% Additionally, models will be created to attempt quality assurance corrections on flagged time periods.
% Although not all failure modes may be able to be successfully addressed (for instance, there is only one major incident of signal noise), at least two should be addressed.
% Measured success of the models can determine which failure modes are the most difficult to accurately repair.

% After a quick read through, a person should feel familiar with your project, and understand the team and the high-level tasks that you will perform. The rest of the document will provide more details on the topics introduced in the summary. In fact, many people find it easiest to write the summary last by pulling key items from the following sections.

% Executive summary max 300 words.

\pagebreak

\section{Literature Review/Market Research}\label{sec_lit}
% This section describes who your research project is for, what they want and what other work has been done on your topic area.. This section is designed to illustrate you have an understanding of the existing research and work previously completed.


\subsection*{Overall research landscape}
% Overall research landscape: Do some basic research about existing research, cite papers and clarify how your project will contribute new knowledge to the world and or utilize novel approaches to existing data sets. 


% 
% In hydrology, machine learning (ML) has been used for predictive runoff modelling,\autocite{mohammadi2021} modelling agricultural drought,\autocite{houmma2022}, MORE EXAMPLES.
Machine learning (ML) has been used in hydrology for applications such as predictive runoff modelling\autocite{mohammadi2021} and modelling agricultural drought.\autocite{houmma2022}
% In hydrology, machine learning (ML) has been used for predictive runoff modelling\autocite{mohammadi2021} and modelling agricultural drought.\autocite{houmma2022}
% Rainfall-runoff data collected by citizen scientists in Ethiopia has been visually (via graphical inspections) and quantitatively (via statistical methods) analyzed for quality compared to nearby professional references.\autocite{mengistiea2024}
% 
% \citean{houmma2022} highlights how incorporation of local and microclimate variables are becoming increasingly necessary for modeling expected droughts due to climate change and ``exceptional situations'' making extreme peaks of variables \& interdependencies no longer able to be statistically modelable.
% 
In the field of remote sensing and geospatial analysis, recurrent~neural~networks~(RNNs)---especially~long short\nobreakdash-term~memory~(LSTM)---are successful for analyzing multi-temporal data, however scalability is limited and they can be prone to overfitting.\autocite{dritsas2025}
% 
In environmental ML, improper missing data management techniques can also worsen data leakage.\autocite{zhu2023}
% Data leakage issues are another consideration in environmental ML, and can stem from improper missing data management techniques.\autocite{zhu2023}

% 
% Rainfall-runoff data collected by citizen scientists in Ethiopia has been visually (via graphical inspections) and quantitatively (via statistical methods) analyzed for quality compared to nearby professional references.\autocite{mengistiea2024}
According to Paton, ML algorithms have yet to be applied to singular catchment run-off data quality assurance.
Previously, stochastic methods were attempted for automation of the Lutz Catchment data quality assurance, however the team concluded it was unsuccessful in being more accurate than Paton's manual corrections.
% However, what makes the Lutz Catchment data unique is th
% Although water flow is a complex phenomenon (especially with other environmental considerations such as soil moisture content and temperature), it is anticipated that the quantity of raw and manually-corrected data and inclusion of additional variables will enable an algorithmic approach to correct the data to a reasonably-accurate degree even without the complexities of hydrological movement.
Although water flow is a complex phenomenon, it is anticipated that the quantity of raw and manually-corrected data and inclusion of additional variables will enable an algorithmic approach to correct the data to a reasonably-accurate degree.

External variables of rainfall and soil moisture content are necessary for a well-informed model.
In the context of modelling expected droughts, \citean{houmma2022} highlights how incorporation of local and microclimate variables are becoming increasingly necessary due to climate change and ``exceptional situations'' which otherwise make extreme peaks of variables \& interdependencies no longer able to be statistically modelable.
% The inclusion of additional variables is also expected to improve the models, as \citean{houmma2022} highlights how incorporation of local and microclimate variables are becoming increasingly necessary for modeling expected droughts due to climate change and ``exceptional situations'' making extreme peaks of variables \& interdependencies no longer able to be statistically modelable.
% 
Rainfall runoff data collected by citizen scientists in Ethiopia has been visually (via graphical inspections) and quantitatively (via statistical methods) analyzed for quality compared to nearby professional references.\autocite{mengistiea2024}
% Previously, stochastic methods were attempted for automation of the Lutz Catchment data quality assurance, however the team concluded it was unsuccessful in being more accurate than Paton's manual corrections.

% In the context of bioinformatics, \citean{kane2022} uses NN workflows for multiclass classification while avoiding data leakage, compensating for missing values, and reducing dimensionality.
% NN workflows are used in bioinformatics for multiclass classification
% The paper focuses on a proposed NN workflow in the context of bioinformatics for multiclass classification while avoiding data leakage, compensating for missing values, and reducing dimensionality.
% To compensate for missing data, kNN-imputation and Iterative imputation were used (numeric models could not be used due to the presence of non-numeric variables). 

% Literature review to show and explain common misunderstandings of ML in environmental research publications.
% Examples from environmental literature of different missing data management (MDM) techniques, including removal, imputation, and replacement (p. 17674). Data leakage issues that can stem from improper MDM are also addressed (p. 17678)


\subsection*{User insights}
% User insights: It is important to listen to potential users to identify their needs and pain points. Utilize the empathy interview to speak with potential users of your research output and list the key insights  and pain points you have discovered during the process and detail how your product will address them. 

% The target user for these models will be hydrology researchers analyzing temporal weir and environmental data in parallel.
% However, given the specificity in location of the weir of interest, it is possible that other factors (such as temperature) may have greater or less influence on run-off data at other weirs around the world and could render final correction models unsuitable for other weirs.
Visual~FoxPro is currently the only software used by Paton to manipulate the data.
While the interface is sufficient for completing the tasks at hand, the software is vulnerable due to its depreciation. Furthermore, it is not currently set up to automatically detect problematic portions of the data.
Because data monitoring is ongoing, new data must be regularly checked for occurrence of any type of failure mode.

% Currently, there is no plan to ``market'' these tools---rather, emphases on readability, accessibility, and reproducibility will be maintained.
% % 

\pagebreak

\section{Research Project Deliverables}\label{sec_deliv}
% This section describes the scope and features of the _analysis_ created by the _project_. It is what the project will PRODUCE. You should write a draft version of these but this is what your mentor will help you with especially. This is often the breakdown of what will and will not be considered a passing project so make this detailed!

% The key objective of the next section is to address the following itels: 
\renewcommand{\thesubsection}{\bf{}}
\setcounter{subsection}{1}
\custsubsection{Final Presentation Fromat}
% Final Presentation Format
% How will your final project be submitted? Will it culminate in a paper with introduction, methods, and results section and at least 5 peer reviewed citations, that is <10 pages long and has at least two informative graphics? Will it culminate in an RShiny application that is also available on GitHub?

The final project will be submitted in the form of a GitHub repository containing code that is relevant, readable, and thoroughly annotated to maximize accessibility and reproducibility.
In addition to code files, a written report with background information, methods/approach, and final model performance statistics \& visualizations will also be created.
% A written report with model performance statistics \& visualizations.

% The system will determine where there are readings containing a potential failure mode. This also requires the system to determine the time range in which a failure mode is occurring.
% A challenge for this feature is that some failures occur in extremely short intervals of time (even as few as one reading), while others can spoil data spanning multiple weeks.

\custsubsection{What Analysis Is Being Run?}
% Once the data is in, what analyses exactly should be run? E.G. what type of neural network is being run, what type of correlation analysis is being run, etc.

\subsubsection*{Classification of Failure Mode}

The system will determine where there are readings containing a potential failure mode. This also requires the system to determine the time range in which a failure mode is occurring.
A challenge for this feature is that some failures occur in extremely short intervals of time (even as few as one reading), while others can spoil data spanning multiple weeks.

% Seasonal autoregressive integrated moving average (seasonal ARIMA [SARIMA]) can be used to analyze time series data that have seasonal patterns by combining seasonal autoregressive (SAR), seasonal differencing (SI), and seasonal moving average (SMA) terms.

Periods of time in the data can be flagged as containing particular ``failure modes'', as explained in Table~\ref{tab_fm}.
To flag data points, supervised classification can utilize rolling statistics and the external environmental variables (which provide some seasonal insights).
Gradient boosted trees or LSTM can be used, but more literature review and exploratory data analysis will be conducted first prior to finalizing which model(s) to utilize.
% Upon a period being flagged as containing a failure mode, it will need to be classified.
% A brief overview of different failure modes is provided in Table~\ref{tab_fm}.

\begin{table}[htbp]
    \centering
    \begin{tabular}{l p{0.395\textwidth} p{0.395\textwidth}}
        \toprule
        Failure Mode & Description & Complicating Factors \\
        \midrule
        Calibration & Standard checks that require baseline correction or slight data pivot. & Recovery after blockage clearing can render calibration points ineffective.\\
        Spike & Short and abrupt changes in level typically caused by equipment issues. & Extremely short-term issue that could be skipped over by random sampling. \\
        Sub-zero & The stream runs dry, or the pond is being drained for cleaning. & Rain during a pond draining can render new data unrecoverable.\\
        Signal noise & Equipment failure results in impossible variability of reported values. & Impossible to manually fix.\\
        % \midrule
        Blockage & Debris blocks the weir's `V', resulting in a gentle increase in water level. & Rainfall events before, during, or shortly after a blockage can interrupt the base flow and decay curves. \\
        % Blockage & Debris blocks the weir's `V', resulting in a gentle increase in water level. \\
        % Blockage after rainfall & A blockage interrupts the normal water level decay curve following a rainfall event. \\
        % Blockage during rainfall & A blockage occurs during a rainfall event, interrupting the base flow and decay curves.\\
        \bottomrule
    \end{tabular}
    \caption{Overview of Weir Failure Modes \parencite{hydroPC}}
    \label{tab_fm}
\end{table}

\subsubsection*{Quality Assurance on Periods of Failure}

% Feature~\hyperref[subsec_class]{2}
Based on the classification result(s), additional models will attempt to adjust the raw outputs to the appropriate values had an error not occurred. This will likely involve creation of multiple separate models, as the method in which the data is corrected differs based on the classification of failure type---for example, an interrupted drainage caused by a blockage in the weir can be fixed by the expected decay curve, while a spike is simply ``flattened'' to the level of adjacent data points.

\custsubsection{What Accuracy Is Expected?}
% Sometimes models don't perform super well. If you're running a model, do you expect 100% accuracy? 65% accuracy across multiple outputs? An R2 of .6? What happens if the model isn't very accurate?

Accuracy for certain failure mode classifications are expected to be high---for instance, ``sub-zero'' values are simple to flag. Others, such as ``spikes'', are also expected to be easily found.
Accuracy for multiple failure modes may be more difficult, especially in the dry season when non-erroneous fluctuations in flow caused by evapotranspiration without rain.

Similarly, correction models for simpler failure modes are expected to be highly accurate and precise, since they are often simple linear gap-fills.
Corrections for the failure mode of a blockage in combination with a rain event may have reasonable accuracy but lower precision.

\custsubsection{What if the Analysis doesn't work?}
% Sometimes analyses don't work. Is a null result acceptable, or do you need to do work to circumvent that?

If the error type categorization has poor performance or is unable to consistently perform, and the needs of the correction models become more pressing as the semester continues, fortunately the informative data set includes the flagged error types.
If correction models do not work, it will still valuable information as it provides insights into which failure modes are most difficult to automate or most ``resistant'' to certain algorithms.

% \custsubsection{What if the Data Isn't Available?}
% This is only for projects that require data scraping – what will happen if your data is not scrapable? How will you salvage your project?


\section{Project Timeline \& Gantt Chart}\label{sec_time}

% In this section, you will list the major deliverables and decision points of the project. You must establish early in the project how you are going to approach the project and get the work done. You will track your work over time using the Gannt Chart and will continually update and submit the chart in weekly check-ins.
% Milestones from the course schedule are pre-populated in the table. You will need to add the dates for the current semester. You will need to add your project specific objectives and deadlines.
% List the MAJOR STEPS needed to complete your project – this includes work done in D1 and D2.  A very common mistake is to build the schedule to have a functional project by D2 Senior Design Day.  This leaves no time for testing and modification (100% guaranteed everything won’t work correctly the 1st time), allow 6-8 weeks for testing minimum).  Also leaves no time to recover if early dates slip.  
% At this stage you should have about 10 – 15 items in addition to the milestones from the Master Schedule. Do not include more milestones from the course master schedule. Create your own meaningful decision or progress checkpoint milestones.

% Some items to consider as examples:
% Determining the sub-system partitioning and assignments
% Preliminary hardware selection
% Selecting a microcontroller/programming language/application framework
% Individual sub-system builds complete
% Individual sub-system testing complete
% Complete system build Integrating the sub-systems together
% Full function testing complete
% Full feature demonstration
% The Project Manager should ensure that the milestones and SOW are reviewed constantly by team members, your D2 mentors your advisor, and the sponsor. 

A tentative schedule of major project milestones is provided in Table~\ref{tab_milestone}, while a detailed Gantt~chart draft is shown in Figure~\ref{fig_gantt}. It is anticipated that integrating the model in combination with managing computational performance will take the greatest amount of time for the project.

\begin{table}[!htbp]
    \centering
    \rowcolors{2}{white}{gray!20}
    \begin{tabular}{l l}
        \toprule
        Milestone & Date \\
        \midrule
        Mentor finalization & 09/12/25 \\
        Project data meeting(s) & 09/19/25 \\
        Bibliography building / lit review & 09/22/25 \\
        Signed proposal & 09/23/25 \\
        EDA & 09/29/25 \\
        Failure mode multiclass categorization model & 10/27/25 \\
        Individual failure mode QA models & 11/24/25 \\
        Final performance measurements & 11/27/25 \\
        % Model instructional writeup & 11/24/25 \\
        % Signed proposal & 2/19/2024 \\
        % Poster Demo & 4/23/24 \\
        iShowcase poster & 12/01/25 \\
        iShowcase presentation practice & 12/04/25 \\
        iShowcase & 12/05/25 \\
        Documentation / repo cleanup & 12/09/25 \\
        Final capstone & 12/10/25 \\
        \bottomrule
    \end{tabular}
    \caption{Milestone Schedule}
    \label{tab_milestone}
\end{table}

% A Gannt chart is a visual representation of the project timeline linked to a schedule with task ownership and dependencies. A Gannt chart is a critical tool in product management. Refer to the Gannt Chart Module on D2L for instructions and resources to complete your Gannt chart.  

% \pagebreak

\begin{figure}[!hp]
% \begin{figure}[!htbp]
    \centering
    % \fbox{\includegraphics[angle=90,width=0.8\linewidth,trim={0 3cm 0 3cm}]{figures/gantt_chart.pdf}}
    \fbox{\includegraphics[angle=90,height=0.8\paperheight,trim={0 3cm 0 3.5cm}]{figures/gantt_chart.pdf}}
    \caption{Gantt Chart Draft}
    % \fbox{\includegraphics[angle=90,width=0.8\linewidth,page=2]{figures/forms.pdf}}
    % \fbox{\includegraphics[width=0.475\linewidth,page=2]{figures/forms.pdf}}
    % \fbox{\includegraphics[width=0.475\linewidth,page=3]{figures/forms.pdf}}
    \label{fig_gantt}
\end{figure}


\pagebreak

\section{Ethics}\label{sec_ethics}

% Data ethics is a key consideration in any product we create. Please include a completed ethics chart. Please see examples on D2L. 

\begin{table}[ht!]
% \begin{table}[htbp]
    \small
    % \singlespacing
    \centering
    \rowcolors{2}{white}{gray!20}
    \renewcommand{\arraystretch}{1.23}
    \begin{tabular}{r p{0.68\textwidth} c c}
        \toprule
        \# & Question & Generally & Data Breach \\
        \midrule
        1 & Could a user sell drugs or other illegal items on your platform? & No & No \\
        2 & Could a user of your platform engage in sex trafficking? & No & No \\
        3 & Could a user sell class notes or cheat on their homework on your platform? & No & No \\
        4 & Could a stalker use your project to find someone? & No & No \\
        5 & Could your app be used to spy on or track individuals? & No & No \\
        6 & Could your app/software access the camera or microphone and record things without users being aware? & No & No \\
        7 & If someone uses your platform, could they be re-traumatized or have their mental health impacted in some way? & No & No \\
        8 & Could your algorithm promote material that would traumatize or upset individuals? & No & No \\
        9 & Would your users be upset if the data you collect was given to someone else? & No & No \\
        10 & Could a data leak potentially lead to identity theft? & No & No \\
        11 & If your site was hacked, would users of that product potentially lose their job, spouse, or family? & No & No \\
        12 & Should there be an age limitation on your product? & No & No \\
        13 & Could someone use your product to find, contact, and potentially commit elder abuse? & No & No \\
        14 & If the data on your platform was breached, could it be used to blackmail the users? & No & No \\
        15 & Does the existence of your project imply that a particular racial group, gender, religion or other protected category is inherently bad, gross, or unwanted? & No & No \\
        16 & Could your product be used to commit hate crimes against a specific group? & No & No \\
        17 & Does the primary content of your game or algorithm focus on something considered deeply unethical? & No & No \\
        18 & Does your game or software contain race, gender, or other stereotypes? & No & No \\
        19 & Could users of your app scam other individuals? & No & No \\
        20 & Is your particular algorithm biased towards predicting correctly only for one race, gender, or other group? & No & No \\
        21 & Are the users of your project, players of your game, or those being surveyed for your data aware of how their data will be used? & No & No \\
        22 & What are the possible misinterpretations of your results? For example: would a white supremacist or misogynist be stoked about your results if they misinterpreted it? & No & No \\
        23 & Does the use or purchase of your data potentially contribute to a dangerous group or regime? & No & No \\
        24 & Could your virtual reality environment cause injury to the user? & No & No \\
        25 & Are your study participants or game players aware that their data will be collected and used? & No & No \\
        26 & Does your game or app contain addictive design elements without benefit to the user? & No & No \\
        27 & Does your survey contain an aspect of compulsion or unusually large incentive, that would command users to take it even if it was to their detriment? & No & No \\
        28 & Could your research outcomes harm an individual or entity? & No & No \\
        \bottomrule
    \end{tabular}
    % \caption{Milestone Schedule}
    \label{tab_ethics}
\end{table}

\pagebreak

\section{Approvals}\label{sec_approvals}

% The project proposal is a document that needs the approval of everyone to move forward. If approval is denied, rework the proposal to fix the issue and begin the approval cycle again. 
% The Faculty advisor should be consulted often for the purposes of creating the proposal. After the Faculty Sponsor and the Team are comfortable with the contents of the SOW, only then should it go to the Sponsor for review. 
% If the Faculty Sponsor and Sponsor are the same person, try to be reasonable regarding what is possible to accomplish.
% Type in the names in the “Approver Name” column, written signatures only in “Signature” and “Date” columns.
% WARNING: Sponsors and Faculty Advisors require lead time to approve documents. Do NOT send a document to them and expect them to read/approve it immediately. Be courteous - give at least 3 business days. Thus, if the SOW is due by 5pm Friday, email it to them on the preceding Tuesday.
% If your sponsor requests an email signature, please talk to your instructor.  It is strongly preferred that they print the signature page, sign, and email a scan of the page.
% Your instructor does not sign before submission, your instructor will sign after grading if the document passes.
% <Keep the language below>


The signatures of the people below indicate an understanding of the purpose and content of this document by those signing it. By signing this document, you indicate that you approve of the proposed project outlined in this Statement of Work, the division of work, the Ground Rules and that the next steps may be taken to create a Product Specification and proceed with the project.
This document is based upon and supersedes the \emph{<PRD title> Version X.X. Deviations}, (versus clarifications), from the PDR have been clearly noted. For any requirements not listed in this SOW, the PRD requirements shall remain in effect.

\begin{table}[htbp]
    \centering
    \renewcommand{\arraystretch}{2}
    \begin{tabular}{lll@{\hskip .55in}l}
    % \begin{tabular}{lll@{\hskip 1.25in}l}
        \toprule
        \toprule
        Approver Name & Title & Signature & Date \\
        \midrule
        \midrule
        % All team members must sign the SOW & Team Member & . & . \\
        % . & Team Member & . & . \\
        % . & Team Member & . & . \\
        Gillian McGinnis & Project Manager & \includegraphics[width=0.2\textwidth]{figures/signature.png} & 09/22/25 \\
        \midrule
        Sriram Iyengar & Primary Advisor &   &   \\
        \midrule
        Steven Paton & Data Advisor &   &   \\
        \midrule
        Nitika Sharma & Course Instructor &   &   \\
        \bottomrule
        \bottomrule
    \end{tabular}
    % \caption{Milestone Schedule}
    \label{tab_approvals}
\end{table}

% The instructor signs the document after it has been graded. 
% Before you submit to advisor/sponsor for review:
% Refresh the table of contents
% Verify every table and figure has a number and caption (right click on the square thing upper left of figure/table and select insert caption).  There should be at least one introductory sentence for every table/figure.
% White space is correct.  Word can be a real pain, but you need to clean up your whitespace (adjust column sizes, make sure page breaks are in proper spot – don’t cut tables in half or put in too many page breaks so that there are ton of mostly blank sheets).
% Make sure the revision block is correct.

% Before you submit for grading – update the author table below. It does not have to be 100% accurate but should be updated if there are major changes and should give the instructor a good idea of how much each person has contributed.
% You can highlight the words in a section to get the word count (it will be displayed in the lower left).  For tables and diagrams just put n/a for the word count.


\begin{table}[htbp]
    \centering
    \begin{tabular}{rllr}
        \toprule
         & Section & Author & Word Count \\
        \midrule
        \secrefboth{sec_es} & Gillian McGinnis & \# \\
        \secrefboth{sec_lit} & \ditto & \# \\
        \secrefboth{sec_deliv} & \ditto & \# \\
        \secrefboth{sec_time} & \ditto & N/A \\
        \secrefboth{sec_ethics} & \ditto & N/A \\
        % \ & \ditto & N/A \\
        \bottomrule
    \end{tabular}
    % \caption{Milestone Schedule}
    \label{tab_contributions}
\end{table}

% \begin{table}[htbp]
%     \centering
%     \begin{tabular}{rllr}
%         \toprule
%          & Section & Author & Word Count \\
%         \midrule
%         % \secrefdot{sec_es} & Executive Summary & Gillian McGinnis & \# \\
%         \secrefboth{sec_es} & \ditto & \# \\
%         \secrefdot{sec_lit} & Literature Review/Market Research & \ditto & \# \\
%         3. & Research Project Deliverables & \ditto & . \\
%         4. & Project Timeline \& Gannt Chart & \ditto & . \\
%         5. & Ethics & \ditto & N/A \\
%         \bottomrule
%     \end{tabular}
%     % \caption{Milestone Schedule}
%     \label{tab_contributions}
% \end{table}

\pagebreak

\section{Appendix}\label{sec_appendix}


\setcounter{subsection}{1}
\renewcommand{\thesubsection}{\Alph{subsection}.\hspace{1em}}
\custsubsection{Advisor Engagement}

\setcounter{subsubsection}{1}
\renewcommand{\thesubsubsection}{\arabic{subsubsection} )\hspace{1em}}
\custsubsubsection{Project Team Responsibilities}
\begin{itemize}
    \item The Project Manager will set up and facilitate a weekly call/meeting with the Faculty Advisor. The Project Team will provide weekly status updates to the Faculty Advisor including upcoming deliverables, critical issues, and any adjustments to the Project Plan.
    \item Documents will be provided to the Faculty Advisor with adequate time for review and signature. The time necessary for review will be agreed with the Advisor. The minimum review time will be 3 days prior to the document due date.
    \item Design files will be provided to the Faculty Advisor as requested in a format agreed to with the Advisor.
    \item Support requirements will be clearly requested from the Faculty Advisor with the dates required and an adequate time for fulfilling the request.
    \item Modifications requests to the Project Plan by Faculty Advisor will be reviewed and agreed to within 1 week of the request.
\end{itemize}

\custsubsubsection{Faculty Advisor Responsibilities}
\begin{itemize}
    \item The Faculty Advisor will provide knowledge and expertise to help the group stretch their skills. 
    \item The Faculty Advisor will participate in a weekly or bi-weekly call/meeting with the Project Team to review the project status, upcoming deliverables, priorities, issues, and progress to the agreed Project Plan.
    \item The Faculty Advisor will provide document review, feedback and approval, rejection, approval with contingencies with adequate time for the Project Team to meet the course due dates. 
    \item The Faculty Advisor will provide feedback to requested support requirements from the Project Team. This includes feedback and guidance on design implementations decisions, design files, test plans, test procedures and test results.
    \item The Faculty Advisor shall provide technical advice and guidance to the Project Team answering inquiries approximately 1 hour per week. 
    \item Modifications to the Project Plan by the Project Team will be resolved and documented within 1 week of the request.
    \item Grade the finalized project using a skill-based rubric
    \item Attend iShowcase in May.
\end{itemize}

\pagebreak
\custsubsection{Ground Rules}
% How the team will conduct the business of completing this project. What are the expectations and ground rules the team will agree to? How will you conduct discussions, manage dissenting views, and make decisions? How will you hold each other accountable for completing this project? Each team member must sign the Approvals section below indicating their acknowledgement of these Ground Rules. Do not remove items from this list, but you may add items to this list.

As a team and as individual team members, we agree to:
\begin{enumerate}%[label=\textbf{\arabic*})
    \item \textbf{Stay focused on our objectives and goals.}\\ Each time the team meets, we will clearly define our objectives and desired outcomes at the beginning of the meeting. We will politely remind team members if we are getting off track.
    \item \textbf{``Sidebar'' any issues that are relevant but not consistent with the immediate objectives.}\\ Occasionally, important matters are raised that are not relevant to the immediate goals of the meeting. To keep the group on track, but avoid losing the issue, create a ``sidebar'' where these topics can be listed and discussed later.
    \item \textbf{Listen when others are speaking.}\\ We will listen and consider others' input before adding our own comments.
    \item \textbf{All viewpoints will have an opportunity to be heard.}\\ We understand that some team members may be quieter than others. We will make an effort to get each team member's viewpoint and that no one dominates the discussion.
    \item \textbf{Differences of opinion will be discussed respectfully.}\\ We will identify areas of agreement before assessing areas of disagreement. We will encourage each other to look beyond our own point of view. We will discuss different ideas respectfully. As a team, we will weigh the merits of different opinions and agree on a process for choosing a direction. All team members will respect and follow the decision or direction.
    \item \textbf{Look for the good points in new ideas.}\\ We will endeavor to explore the value in each idea as we assess and select our path forward.
    \item \textbf{Focus on the future, not the past.}\\ We will use our past experience to inform our decisions, but focus the discussion on the future objectives. Blame for past performance is counterproductive, we will focus on finding solutions.
    \item \textbf{Agree upon specific action items and next steps.}\\ At the end of each meeting and discussion, we will summarize and agree on specific next steps, action items and assignments.
    \item \textbf{Accountability}\\ As team members, we will each be responsible for our individual assignments and contribution to achieving the team objectives and goals. We will honor our responsibilities and not let our team members down.
\end{enumerate}

\pagebreak
% \nocite{*}

\printbibliography
\label{sec_bib}

\end{document}