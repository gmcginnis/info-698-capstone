\documentclass[11pt, letterpaper]{article}

% Compact lists
\usepackage{enumitem}

% Images
\usepackage{graphicx}

% Tables
\usepackage{booktabs}

% Document boarders
\usepackage{geometry}
\geometry{
    letterpaper,
    margin=.85in
}

% Remove page numbers
% \pagenumbering{gobble}

% % Double spacing
% \usepackage{setspace}
% \doublespacing

% Spacing out footnotes
\setlength{\footnotesep}{0.75\baselineskip}

% Palatino
\usepackage{newpxtext}
\usepackage{newpxmath}

% Nicer monospace font
\usepackage{inconsolata}

% Improved typography
\usepackage{microtype}

% URL and hyperlink settings
\usepackage[dvipsnames]{xcolor}
\usepackage{xurl}
\usepackage[
    colorlinks,
    breaklinks=true,
    urlcolor=NavyBlue,
    citecolor=NavyBlue,
    linkcolor=NavyBlue,
    % breaklinks,
    linktocpage
]{hyperref}
\urlstyle{same}

% Bib settings
\usepackage[
    backend=biber,
    % style=apa,
    style=chem-acs,
    doi=true,
    % url=true,
    url=false,
    articletitle=true,
    maxbibnames=99
    %natbib=true
    ]{biblatex}
\addbibresource{bibliography.bib}

% % ToC Settings
% \usepackage{titlesec}
% \titleformat{\subsection}{\normalfont\Large}{Feature\ \thesubsection :}{0.5ex}{}
% \titlelabel{\thesubsubsection}
% \usepackage{tocstyle}
\usepackage{tocloft}
% \newlength{\origsubsecwidth}
% \setlength{\origsubsecwidth}{\cftsubsecnumwidth}

% Add dots for sections
\renewcommand{\cftsecleader}{\cftdotfill{\cftdotsep}}
% Remove dots for subsections
\renewcommand{\cftsubsecdotsep}{\cftnodots}
\renewcommand{\cftsubsubsecdotsep}{\cftnodots}

% \addtolength{\cftsubsecnumwidth}{25pt}
% \addtolength{\cftsubsecnumwidth}{-25pt}
\setlength{\cftsubsecnumwidth}{0pt}
\setlength{\cftsubsubsecnumwidth}{0pt}

\newcommand{\custsubsection}[1]{%
    \subsection*{\thesubsection {#1}} %
    \addcontentsline{toc}{subsection}{\protect\numberline{}{\thesubsection #1}} %
    \addtocounter{subsection}{1}
}

\newcommand{\custsubsubsection}[1]{%
    \subsubsection*{\thesubsubsection {#1}} %
    \addcontentsline{toc}{subsubsection}{\protect\numberline{}{\thesubsubsection #1}} %
    \addtocounter{subsubsection}{1}
}

\begin{document}

\begin{titlepage}
    \begin{center}
        
        \Huge
        Your Project Name

        \LARGE
        Project proposal \& Statement of Work
        \break
        PM's name, Project Manager
        
        List remaining team members here, one name per line, first name, last name, alphabetically by last name
        
        POTENTIAL ADVISORS: \emph{LIST POTENTIAL ADVISORS FOR GROUP HERE}
        
        Date: Month, Day, Year
    \end{center}
\end{titlepage}

% Use the statements in red italics to help you write this document. Delete all the red italics when you are done.

% This document is both a project proposal and statement of work document.  It states, “Here’s the problem we have identified and this is our plan to develop a solution.”  It also clearly details what you will require from the mentor to be successful. It is the contract between you, and the mentor which summarizes who will do what and the requirements for success.

% SOW: The Product Specification defines your product in detail. What you are making, how it works, how it is used and how it will be tested.

% Each section shall have a named author, but all team members are responsible for the overall document.  Meaning one team member writes the section and the other team members review it for errors.  List the author of each section immediately following the heading and in the author table in the grading rubric at the end. Example: “This section was written by John Doe.”  It is the team and each members’ responsibility to divide the document equitably. Individual scores will be determined by the quality and quantity of each person’s work.

% Remember the goal is to convey all the information – not applicable because…. is relevant information.  Use bulleted lists and white space smartly. Include figures and tables as required and necessary. For each table and figure, write a brief introductory sentence “As shown in Table 2 five different pages are used…”  Put the detail in the figures and tables, keep the paragraphs short and do not duplicate information needlessly.  Your goal is to convey the full amount of information in the shortest possible document. 

% You need to maintain a Revision History Table like this: (Not this exact table with ONLY the dates changed)

% \begin{table}[!ht]
%     \centering
%     \caption{An example of the raw spectra data. The ``Variety'' column is not unique for the dataset, and can be disregarded during analysis.}
%     \label{tab_revision}
%     \begin{tabular}{r r r} 
%         \toprule
%         Version & Summary of Changes & Date
%         \midrule
%         V5 & Added a ``Ground Rules'' section & 6/15/22 \\
%         V7 & Added Sponsor and ADvisor Requirements & 8/17/22
%         \bottomrule
%     \end{tabular}
% \end{table}

% The revision history table documents major revisions to the document. In addition to showing the evolution of the project it also provides a way for reviewers to quickly see which sections have changed so they do not have to reread the entire document each time. It is a high-level summary, minor edits, spelling corrections, etc. should not need to be summarized in this block. It is your responsibility to determine which sections have significant changes that merit review and decide when it makes sense to declare a revision number (hint: call a team meeting to review it page by page). Less detail is required for early revisions.  Typically, teams use collaborative tools like Google Docs or Office 365 to allow real-time editing. There may be hundreds of changes made in a few days; the revision block should reflect only the high-level changes. E.g., “Section 2.1 was updated to reflect the switch to the xyz processor.” For a shorter document such as the SOW you may only have 2 or 3 versions.  If you do update it later be sure to include the revision information.

% The table of contents below is auto generated – as long as you use Heading 1 and Heading 2 styles you should not have to type anything into the table.  You will have to refresh it.  The pagination on this template is not set, use your best judgement on when to start a new page for clarity balanced with keeping excessive white space out of your document.  A suggestion is to have the revision block and TOC on one page, insert a page break before the Executive Summary, then do what make sense and looks professional after that.


\tableofcontents

\pagebreak

How does this work \parencite{kane2022}?

\section{Executive Summary}
% The Executive Summary was written by John Doe.

% (Complete this with the author of each section)
% The executive summary is a short review of the important aspects of the project.  It is very similar to the elevator pitch, but it is slightly longer with more detail. This section should be about 3 paragraphs and about ¾ of a page.
% 1st paragraph - Introduce your product - What is your product and its key features? (“Our product [is a high efficiency controller/web-based application/wireless smart sensor/….] that [displays/computes/transforms/…] [temperature data/DC current/] into [easy to read graphs/control signals/wireless data] while [lowering cost/improving response time/achieving high efficiency/].
% 2nd paragraph - Explain why the product is needed and what problem it is solving. If current solutions to the problem exists explain how your solution is better and unique. Some context is needed here but only enough to establish “Why?” product will benefit the user.) If it is an internal project, state why it is important. This may be to facilitate learning, but hopefully it also has a tie to some greater good for society. Find credible sources and cite them in a footnote. Don’t say “everyone knows flooding is an issue…” research and say, “according to <insert credible reference, peer reviewed paper, gov’t agency, etc> flooding in 2019 caused $32M in damages in Acme County, our product will aid in the detection….”
% 3rd Paragraph - Describe the process of development: What work will you be doing (“We will be /building XYZ) Who is involved, Where the project work will take place, what you will have completed at the end of semester.
% Include a table as shown. For this and all tables you do not want to repeat the information in the table – that’s what the table is for.  Rather you should introduce the table and point out anything significant. Something like “Table 1 shows the preliminary division of responsibilities among the team members.  Since the User Interface for this product is quite complex, we have divided it among multiple team members as listed.”

\begin{table}[htbp]
    \centering
    \begin{tabular}{l|l}
        \toprule
        Team Member & Feature responsibility \\
        \midrule
        Violet Parr & UI: Display \\
        Helen Parr & Remote mgmt \\
        \bottomrule
    \end{tabular}
    \caption{Preliminary Subsystem Responsibilities}
    \label{tab_sample}
\end{table}

% After a quick read through, a person should feel familiar with your project, and understand the team and the high-level tasks that you will perform. The rest of the document will provide more details on the topics introduced in the summary. In fact, many people find it easiest to write the summary last by pulling key items from the following sections.

% Executive summary max 300 words.

\section{User/Market Research}
% This section describes who your product is for and what they want. One of the worst things product developers can do is create a product that no one wants, needs or will pay for. This section is designed to illustrate you have an understanding of your user, the market for your product and should be updated frequently as you learn more. In this section you should look at the macro market to evaluate existing products and market opportunity. You should also look at the micro market by performing empathy interviews with potential users. Please refer to the research module for more information. 

% Overall market: Do some basic research about the size and scope of the market your product will be in. If it is a web-based game please include the size and scope of web based games. How many players, how many games, how much revenue is created in the market. 

% Existing competitors: List your closest competitors and any information you can glean about their market including user base, etc. Also include how your product will be differentiated from them. 

% User insights: It is important to listen to potential users to identify their needs and pain points. Utilize the empathy interview to speak with potential users and list the key insights  and pain points you have discovered during the process and detail how your product will address them. 

\section{Product Features}
% This section describes the scope and features of the _product_ created by the _project_. It is what the project will PRODUCE.. 

% The key objective of the next section is to: 
% •	list all the key features of the product “What does it do?” and who will do them. Be sure the product includes enough work to fulfill the requirements of the class, but not so many as to create a recipe for failure. I suggest your include a minimum feature list and a stretch feature list prioritized by functionality. 
% •	summarize the parameters for each feature 
% •	briefly elaborate on the relationship between the product and the business need it addresses. “Why does the user want this feature?  What benefit does it provide?” 
% The product features are the major functional items that the team agrees they will accomplish.  Features typically describe what a user of the product will interact with or observe about the product. Once all the items are completed it will represent the final project. This will be one of the criteria for judging the success of the project. It is very important to limit the scope to only what you can accomplish. It also protects the team from other people adding new product features or functions later (the dreaded “feature creep”).  Use tables and bulleted lists in this section to keep it concise.
% A feature is an item that would be listed on the sales pitch to get a customer interested in buying the product. This is not a sales pitch to Engineers; it is a description of what the product accomplishes. i.e., a feature of a display for User interface. Not a 7-inch OLED 24-bit 1080x720P display.

% \renewcommand{\thesubsection}{Feature \arabic{subsection}:}
% \renewcommand{\thesubsection{}}
% \renewcommand{\thesubsection}{\Alph{subsection}}

% \renewcommand{\numberline}[1]{}
% \subsection{Wireless Transmission of Data}

\renewcommand{\thesubsection}{Feature \arabic{subsection}.\ }
\setcounter{subsection}{1}
% \subsection*{Feature \thesubsection : Wireless Transmission of Data}
% \addcontentsline{toc}{subsection}{\protect\numberline{}Feature \thesubsection : Wireless Transmission of Data}
\custsubsection{Wireless Transmission of Data}

% \custsubsection{Feature \thesubsection : Againagain}
\custsubsection{Again}
% \custsubsection{Feature \thesubsection : Blah}

% \addcontentsline{toc}{subsection}{\protect\numberline{Feature \thesubsection} : Wireless Transmission of Data}
% \addcontentsline{toc}{subsection}{\protect\numberline{Feature \thesubsection : Wireless Transmission of Data}}
% \addcontentsline{toc}{subsection}{\protect\numberline{Feature \thesubsection} : Wireless Transmission of Data}


% \renewcommand{\thesubsection}{\Alph{subsection}}
% \renewcommand{\thesubsubsection}{\arabic{subsubsection}\\)}
% \subsection{Feature 1: Wireless Transmission of Data}

% \subsection{Feature \thesubsection : Another section}
% \subsection{Another}
% \addcontentsline{toc}{subsection}{\protect\numberline{}Feature \thesubsection : Another}

% This section was written by (the owner of the subsystem that implements this feature should write this section.)
% Write a sentence to briefly describe the feature and introduce the table of parameters. Think hard about how to define the requirements – usually min and max measurable values of some sort. Example:

% The system will transmit the water level readings from the remote unit located at the retention pond to the main unit located in the construction office.  The requirements for transmission are summarized in Table 1.

\section{Project Timeline \& Gannt Chart}

% In this section, you will list the major deliverables and decision points of the project. You must establish early in the project how you are going to approach the project and get the work done. You will track your work over time using the Gannt Chart and will continually update and submit the chart in weekly check-ins.

% Milestones from the course schedule are pre-populated in the table. You will need to add the dates for the current semester. You will need to add your project specific objectives and deadlines.

% List the MAJOR STEPS needed to complete your project – this includes work done in D1 and D2.  A very common mistake is to build the schedule to have a functional project by D2 Senior Design Day.  This leaves no time for testing and modification (100\% guaranteed everything won’t work correctly the 1st time), allow 6-8 weeks for testing minimum).  Also leaves no time to recover if early dates slip.  

% At this stage you should have about 10 – 15 items in addition to the milestones from the Master Schedule. Do not include more milestones from the course master schedule. Create your own meaningful decision or progress checkpoint milestones.

\begin{table}[htbp]
    \centering
    \begin{tabular}{l|l}
        \toprule
        Milestone & Date \\
        \midrule
        Team formulation & 1/29/24 \\
        . & . \\
        . & . \\
        Signed proposal & 2/19/2024 \\
        . & . \\
        Poster Demo & 4/23/24 \\
        . & . \\
        iShowcase & 5/1/25 \\
        \bottomrule
    \end{tabular}
    \caption{Milestone Schedule}
    \label{tab_milestone}
\end{table}

% Some items to consider as examples:
% Determining the sub-system partitioning and assignments
% Preliminary hardware selection
% Selecting a microcontroller/programming language/application framework
% Individual sub-system builds complete
% Individual sub-system testing complete
% Complete system build Integrating the sub-systems together
% Full function testing complete
% Full feature demonstration
% The Project Manager should ensure that the milestones and SOW are reviewed constantly by team members, your D2 mentors your advisor, and the sponsor. 
% A Gannt chart is a visual representation of the project timeline linked to a schedule with task ownership and dependencies. A Gannt chart is a critical tool in product management. Refer to the Gannt Chart Module on D2L for instructions and resources to complete your Gannt chart.  

\section{Ethics}

% Data ethics is a key consideration in any product we create. Please include a completed ethics chart. Please see examples on D2L. 

\begin{table}[htbp]
    \centering
    \begin{tabular}{r|l|c|c}
        \toprule
        \# & Question & Generally & Data Breach \\
        \midrule
        1 & Could a user sell drugs or other illegal items on your platform? & Y/N/M & Y/N/M \\
        . & . & . & .\\
        28 & Could your research outcomes harm an individual or entity? & Y/N/M & Y/N/M \\
        \bottomrule
    \end{tabular}
    % \caption{Milestone Schedule}
    \label{tab_ethics}
\end{table}

\section{Approvals}

% The project proposal is a document that needs the approval of everyone to move forward. If approval is denied, rework the proposal to fix the issue and begin the approval cycle again. 
% The Faculty advisor should be consulted often for the purposes of creating the proposal. After the Faculty Sponsor and the Team are comfortable with the contents of the SOW, only then should it go to the Sponsor for review. 
% If the Faculty Sponsor and Sponsor are the same person, try to be reasonable regarding what is possible to accomplish.
% Type in the names in the “Approver Name” column, written signatures only in “Signature” and “Date” columns.
% WARNING: Sponsors and Faculty Advisors require lead time to approve documents. Do NOT send a document to them and expect them to read/approve it immediately. Be courteous - give at least 3 business days. Thus, if the SOW is due by 5pm Friday, email it to them on the preceding Tuesday.
% If your sponsor requests an email signature, please talk to your instructor.  It is strongly preferred that they print the signature page, sign, and email a scan of the page.
% Your instructor does not sign before submission, your instructor will sign after grading if the document passes.
% <Keep the language below>

The signatures of the people below indicate an understanding of the purpose and content of this document by those signing it. By signing this document, you indicate that you approve of the proposed project outlined in this Statement of Work, the division of work, the Ground Rules and that the next steps may be taken to create a Product Specification and proceed with the project.
This document is based upon and supersedes the \emph{<PRD title> Version X.X. Deviations}, (versus clarifications), from the PDR have been clearly noted. For any requirements not listed in this SOW, the PRD requirements shall remain in effect.

\begin{table}[htbp]
    \centering
    \begin{tabular}{l|l|c|l}
        \toprule
        Approver Name & Title & Signature & Date \\
        \midrule
        All team members must sign the SOW & Team Member & . & . \\
        . & Team Member & . & . \\
        . & Team Member & . & . \\
        . & Project Manager & . & . \\
        . & Advisor & . & . \\
        . & Instructor & . & . \\
        \bottomrule
    \end{tabular}
    % \caption{Milestone Schedule}
    \label{tab_approvals}
\end{table}

% The instructor signs the document after it has been graded. 
% Before you submit to advisor/sponsor for review:
% Refresh the table of contents
% Verify every table and figure has a number and caption (right click on the square thing upper left of figure/table and select insert caption).  There should be at least one introductory sentence for every table/figure.
% White space is correct.  Word can be a real pain, but you need to clean up your whitespace (adjust column sizes, make sure page breaks are in proper spot – don’t cut tables in half or put in too many page breaks so that there are ton of mostly blank sheets).
% Make sure the revision block is correct.

% Before you submit for grading – update the author table below. It does not have to be 100% accurate but should be updated if there are major changes and should give the instructor a good idea of how much each person has contributed.
% You can highlight the words in a section to get the word count (it will be displayed in the lower left).  For tables and diagrams just put n/a for the word count.

\begin{table}[htbp]
    \centering
    \begin{tabular}{l|l|r}
        \toprule
        Section & Author & Word Count \\
        \midrule
        1. Introduction & Jane Doe & 268 \\
        . & . & . \\
        . & . & . \\
        . & . & . \\
        \bottomrule
    \end{tabular}
    % \caption{Milestone Schedule}
    \label{tab_contributions}
\end{table}

\pagebreak

\section{Appendix}

% \setlength{\cftsubsecnumwidth}{\origsubsecwidth}
% \addtolength{\cftsubsecnumwidth}{-25pt}
% \renewcommand{\thesubsection}{\Alph{subsection}}

% \renewcommand{\thesubsection}{Appendix \Alph{subsection}}
% \renewcommand{\thesubsubsection}{\arabic{subsubsection}}

\setcounter{subsection}{1}
\renewcommand{\thesubsection}{\Alph{subsection}.\hspace{1em}}
\custsubsection{Advisor Engagement}

% \renewcommand{\addtolength{\cftsubsecnumwidth}{0pt}}
% \subsection*{Feature \thesubsection : Wireless Transmission of Data}
% \addcontentsline{toc}{subsection}{\protect\numberline{}Feature \thesubsection : Wireless Transmission of Data}
% \subsection{Advisor Engagement}

\setcounter{subsubsection}{1}
\renewcommand{\thesubsubsection}{\arabic{subsubsection} )\hspace{1em}}
% \subsubsection{Project Team Responsibilities}
\custsubsubsection{Project Team Responsibilities}
\begin{itemize}
    \item The Project Manager will set up and facilitate a weekly call/meeting with the Faculty Advisor. The Project Team will provide weekly status updates to the Faculty Advisor including upcoming deliverables, critical issues, and any adjustments to the Project Plan.
    \item Documents will be provided to the Faculty Advisor with adequate time for review and signature. The time necessary for review will be agreed with the Advisor. The minimum review time will be 3 days prior to the document due date.
    \item Design files will be provided to the Faculty Advisor as requested in a format agreed to with the Advisor.
    \item Support requirements will be clearly requested from the Faculty Advisor with the dates required and an adequate time for fulfilling the request.
    \item Modifications requests to the Project Plan by Faculty Advisor will be reviewed and agreed to within 1 week of the request.
\end{itemize}

\custsubsubsection{Faculty Advisor Responsibilities}
\begin{itemize}
    \item The Faculty Advisor will provide knowledge and expertise to help the group stretch their skills. 
    \item The Faculty Advisor will participate in a weekly or bi-weekly call/meeting with the Project Team to review the project status, upcoming deliverables, priorities, issues, and progress to the agreed Project Plan.
    \item The Faculty Advisor will provide document review, feedback and approval, rejection, approval with contingencies with adequate time for the Project Team to meet the course due dates. 
    \item The Faculty Advisor will provide feedback to requested support requirements from the Project Team. This includes feedback and guidance on design implementations decisions, design files, test plans, test procedures and test results.
    \item The Faculty Advisor shall provide technical advice and guidance to the Project Team answering inquiries approximately 1 hour per week. 
    \item Modifications to the Project Plan by the Project Team will be resolved and documented within 1 week of the request.
    \item Grade the finalized project using a skill-based rubric
    \item Attend iShowcase in May.
\end{itemize}

% \subsection{Ground Rules}
\custsubsection{Ground Rules}
% How the team will conduct the business of completing this project. What are the expectations and ground rules the team will agree to? How will you conduct discussions, manage dissenting views, and make decisions? How will you hold each other accountable for completing this project? Each team member must sign the Approvals section below indicating their acknowledgement of these Ground Rules. Do not remove items from this list, but you may add items to this list.

As a team and as individual team members, we agree to:
\begin{enumerate}%[label=\textbf{\arabic*})
    \item \textbf{Stay focused on our objectives and goals.}\\ Each time the team meets, we will clearly define our objectives and desired outcomes at the beginning of the meeting. We will politely remind team members if we are getting off track.
    \item \textbf{``Sidebar'' any issues that are relevant but not consistent with the immediate objectives.}\\ Occasionally, important matters are raised that are not relevant to the immediate goals of the meeting. To keep the group on track, but avoid losing the issue, create a ``sidebar'' where these topics can be listed and discussed later.
    \item \textbf{Listen when others are speaking.}\\ We will listen and consider others' input before adding our own comments.
    \item \textbf{All viewpoints will have an opportunity to be heard.}\\ We understand that some team members may be quieter than others. We will make an effort to get each team member's viewpoint and that no one dominates the discussion.
    \item \textbf{Differences of opinion will be discussed respectfully.}\\ We will identify areas of agreement before assessing areas of disagreement. We will encourage each other to look beyond our own point of view. We will discuss different ideas respectfully. As a team, we will weigh the merits of different opinions and agree on a process for choosing a direction. All team members will respect and follow the decision or direction.
    \item \textbf{Look for the good points in new ideas.}\\ We will endeavor to explore the value in each idea as we assess and select our path forward.
    \item \textbf{Focus on the future, not the past.}\\ We will use our past experience to inform our decisions, but focus the discussion on the future objectives. Blame for past performance is counterproductive, we will focus on finding solutions.
    \item \textbf{Agree upon specific action items and next steps.}\\ At the end of each meeting and discussion, we will summarize and agree on specific next steps, action items and assignments.
    \item \textbf{Accountability}\\ As team members, we will each be responsible for our individual assignments and contribution to achieving the team objectives and goals. We will honor our responsibilities and not let our team members down.
\end{enumerate}

\pagebreak

\printbibliography

\end{document}